\newcommand{\bushere}[3]{% length, text above, text below}
% Optional arguments do nto work in paths
%
% starting point; draw an edge and then two nodes
% save the position
coordinate(tmp)
% go up and do an edge down
++(0,#1) node[anchor=base, font=\footnotesize]{#2} edge[ultra thick] ++(0, {-2*#1})
% edges do not move the current point, go down to position the node
++(0,{-2*#1}) node[below]{#3}
% go back to where we started
(tmp)
}

\ctikzset{sources/fill=gray!20, resistors/fill=gray!20}
\resizebox{\linewidth}{!}{ % Set width to \linewidth
	\begin{tikzpicture}[semithick]% default line width
		% Buses and branches
		\draw (0,0)
		node[left, font=\footnotesize]{Generator} ++(1.5,0) \bushere{1.5}{Connection Point}{} coordinate(poc);
		\draw (poc)
		-- ++ (1,0) to[generic, l={$Z_{\text{grid}}$}, resistors/width=2] ++ (4,0)
		-- ++ (1,0)
		\bushere{1.5}{Infinite Bus}{} coordinate(infinite bus);
		% One load (start from the coord load, go up)
		\ctikzset{bipoles/border margin=0.5}% See manual section 3.1.2
		\draw (infinite bus) -- ++(1,0) node[vsourcesinshape, rotate=90]{} coordinate(vgrid) ++(0.5,0) node[right]{$V_{\text{grid}}$};
		\draw (poc) -- ++(-1,0) node[vsourcesinshape, rotate=90]{};
		% Fault
		\draw[blue, font=\footnotesize, <-] (vgrid) ++(0, -0.5) -- ++(0, -0.5) -- ++(0.5,0) node[right]{$V_{\mathrm{grid}_{\mathrm{1}}}, V_{\mathrm{grid}_{\mathrm{2}}}, V_{\mathrm{grid}_{\mathrm{3}}}, \dots, V_{\mathrm{grid}_{\mathrm{n}}}$};
		
	\end{tikzpicture}
}





