Temporary Over-Voltage tests assess the ability of the generator to ride through high voltage disturbances and supply the correct amount of inductive reactive current. To perform these tests, the appropriate $V_{\mathrm{grid}_{\mathrm{initial}}}$
is first identified to achieve $V_{\mathrm{POC}_{\mathrm{initial}}}$ given the required initial $P_{\mathrm{POC}}$, $Q_{\mathrm{POC}}$, SCR and X/R conditions. 

A shunt capacitor is then inserted at the Connection Point, sized such that $V_{\mathrm{POC}_{\mathrm{TOV}}} = k_{\mathrm{OV}} * V_{\mathrm{POC}_{\mathrm{initial}}}$, where $k_{\mathrm{OV}}$ is the desired percentage increase in $V_{\mathrm{POC}}$.

The test is then performed by initialising the system with the shunt capacitor out of service, then switching it in for the intended disturbance duration, as shown in Figure~\ref{fig:smib-tov-diagram}.

It should be noted that due to the dynamics of capacitor switching, the initial instantaneous voltage spike may appear filtered and not reach $k_{\mathrm{OV}} * V_{\mathrm{POC}_{\mathrm{initial}}}$. The settled $P_{\mathrm{POC}}$ will typically also be lower than this value due to the inductive reactive current support of the generator.

\begin{figure}[h]
	\centering
	\newcommand{\bushere}[3]{% length, text above, text below}
% Optional arguments do nto work in paths
%
% starting point; draw an edge and then two nodes
% save the position
coordinate(tmp)
% go up and do an edge down
++(0,#1) node[anchor=base, font=\footnotesize]{#2} edge[ultra thick] ++(0, {-2*#1})
% edges do not move the current point, go down to position the node
++(0,{-2*#1}) node[below]{#3}
% go back to where we started
(tmp)
}

\ctikzset{sources/fill=gray!20, resistors/fill=gray!20}
\resizebox{0.65\linewidth}{!}{ % Set width to \linewidth
	\begin{tikzpicture}[semithick]% default line width
		% Buses and branches
		\draw (0,0)
		node[left, font=\footnotesize]{Generator} ++(1.5,0) \bushere{1.5}{Connection Point}{} coordinate(poc);
		\draw (poc)
		-- ++ (1,0) to[generic, l={$Z_{\text{grid}}$}, resistors/width=2] ++ (4,0)
		-- ++ (1,0)
		\bushere{1.5}{Infinite Bus}{} coordinate(infinite bus);
		% One load (start from the coord load, go up)
		\ctikzset{bipoles/border margin=0.5}% See manual section 3.1.2
		\draw (infinite bus) -- ++(1,0) node[vsourcesinshape, rotate=90]{} ++(0.5,0) node[right]{$V_{\text{grid}}$};
		\draw (poc) -- ++(-1,0) node[vsourcesinshape, rotate=90]{};
		% Capacitor
		\draw[red](poc) ++(0, -0.7) -- ++(0.7,0) -- ++(0,-0.5) to[C] ++(0,-1) node[ground]{};
		
	\end{tikzpicture}
}






	\caption{TOV test application methodology}
	\label{fig:smib-tov-diagram}
\end{figure}