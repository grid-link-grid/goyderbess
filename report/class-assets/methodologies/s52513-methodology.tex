Reactive controller stability tests assess the ability of the generator to provide stable reactive response to a perturbed Connection Point voltage or controller state (i.e. a reference change). In VAr and power factor control modes, this is just about  $P_{\mathrm{POC}}$ and $Q_{\mathrm{POC}}$ settling to their pre-disturbance values. However, in voltage droop control modes, a changing $V_{\mathrm{POC}}$ will result in a changing calculated $Q_{\mathrm{ref}}$, so the generator will need to track to a new reactive power target at the same time as rejecting the disturbance.

To implement these tests in \ac{SMIB} modelling, the appropriate $V_{\mathrm{grid}_{\mathrm{1}}}$
is first identified to achieve $V_{\mathrm{POC}_{\mathrm{1}}}$ given the required initial $P_{\mathrm{POC}}$, $Q_{\mathrm{POC}}$, SCR and X/R conditions. Subsequent $V_{\mathrm{grid}}$ values $V_{\mathrm{grid}_{\mathrm{2}}}, V_{\mathrm{grid}_{\mathrm{3}}}, \dots, V_{\mathrm{grid}_{\mathrm{n}}}$ can then be calculated to achieve the desired $V_{\mathrm{POC}}$ values $V_{\mathrm{POC}_{\mathrm{2}}}, V_{\mathrm{POC}_{\mathrm{3}}}, \dots, V_{\mathrm{POC}_{\mathrm{n}}}$.

With all $V_{\mathrm{grid,i}}$ calculated, a simulation is performed with $V_{\mathrm{grid}}$ stepped or ramped as required to implement the desired disturbance, as shown in Figure~\ref{fig:smib-vgrid-disturbance-diagram}.

\begin{figure}[H]
	\centering
	\newcommand{\bushere}[3]{% length, text above, text below}
% Optional arguments do nto work in paths
%
% starting point; draw an edge and then two nodes
% save the position
coordinate(tmp)
% go up and do an edge down
++(0,#1) node[anchor=base, font=\footnotesize]{#2} edge[ultra thick] ++(0, {-2*#1})
% edges do not move the current point, go down to position the node
++(0,{-2*#1}) node[below]{#3}
% go back to where we started
(tmp)
}

\ctikzset{sources/fill=gray!20, resistors/fill=gray!20}
\resizebox{\linewidth}{!}{ % Set width to \linewidth
	\begin{tikzpicture}[semithick]% default line width
		% Buses and branches
		\draw (0,0)
		node[left, font=\footnotesize]{Generator} ++(1.5,0) \bushere{1.5}{Connection Point}{} coordinate(poc);
		\draw (poc)
		-- ++ (1,0) to[generic, l={$Z_{\text{grid}}$}, resistors/width=2] ++ (4,0)
		-- ++ (1,0)
		\bushere{1.5}{Infinite Bus}{} coordinate(infinite bus);
		% One load (start from the coord load, go up)
		\ctikzset{bipoles/border margin=0.5}% See manual section 3.1.2
		\draw (infinite bus) -- ++(1,0) node[vsourcesinshape, rotate=90]{} coordinate(vgrid) ++(0.5,0) node[right]{$V_{\text{grid}}$};
		\draw (poc) -- ++(-1,0) node[vsourcesinshape, rotate=90]{};
		% Fault
		\draw[blue, font=\footnotesize, <-] (vgrid) ++(0, -0.5) -- ++(0, -0.5) -- ++(0.5,0) node[right]{$V_{\mathrm{grid}_{\mathrm{1}}}, V_{\mathrm{grid}_{\mathrm{2}}}, V_{\mathrm{grid}_{\mathrm{3}}}, \dots, V_{\mathrm{grid}_{\mathrm{n}}}$};
		
	\end{tikzpicture}
}






	\caption{Reactive power controller disturbance assessment methodology}
	\label{fig:smib-vgrid-disturbance-diagram}
\end{figure}

To assess the stability of the control system response to a stepped $V_{\mathrm{ref}}$, $Q_{\mathrm{ref}}$, $PF_{\mathrm{ref}}$ signal, reference steps are applied as per Figure \ref{fig:smib-vref-change-diagram}. To perform this test, the generator is first initialised to the initial $V_{\mathrm{POC}}$, $P_{\mathrm{POC}}$, $Q_{\mathrm{POC}}$, SCR and X/R conditions, where $Q_{\mathrm{POC}}$ is the target reactive output of the generator for the associated $V_{\mathrm{err}} = V_{\mathrm{ref}_{\mathrm{1}}} - V_{\mathrm{POC}}$ per the droop characteristic.

Once the generator has been initialised, the series of voltage references $V_{\mathrm{ref}_{\mathrm{2}}}, V_{\mathrm{ref}_{\mathrm{3}}}, \dots, V_{\mathrm{ref}_{\mathrm{n}}}$ are applied to the PPC, as shown in Figure~\ref{fig:smib-vref-change-diagram}. Tests where the reactive power reference is constrained by a reactive power limit, will be identified as "Saturating".

\begin{figure}[H]
	\centering
	\newcommand{\bushere}[3]{% length, text above, text below}
% Optional arguments do nto work in paths
%
% starting point; draw an edge and then two nodes
% save the position
coordinate(tmp)
% go up and do an edge down
++(0,#1) node[anchor=base, font=\footnotesize]{#2} edge[ultra thick] ++(0, {-2*#1})
% edges do not move the current point, go down to position the node
++(0,{-2*#1}) node[below]{#3}
% go back to where we started
(tmp)
}

\ctikzset{sources/fill=gray!20, resistors/fill=gray!20}
\resizebox{\linewidth}{!}{ % Set width to \linewidth
	\begin{tikzpicture}[semithick]% default line width
		% Buses and branches
		\draw (0,0)
		node[left, font=\footnotesize]{Generator} ++(1.5,0) \bushere{1.5}{Connection Point}{} coordinate(poc);
		\draw (poc)
		-- ++ (1,0) to[generic, l={$Z_{\text{grid}}$}, resistors/width=2] ++ (4,0)
		-- ++ (1,0)
		\bushere{1.5}{Infinite Bus}{} coordinate(infinite bus);
		% One load (start from the coord load, go up)
		\ctikzset{bipoles/border margin=0.5}% See manual section 3.1.2
		\draw (infinite bus) -- ++(1,0) node[vsourcesinshape, rotate=90]{} coordinate(vgrid) ++(0.5,0) node[right]{$V_{\text{grid}}$};
		\draw (poc) -- ++(-1,0) node[vsourcesinshape, rotate=90]{} coordinate(generator);
		% Fault
		\draw[blue, font=\footnotesize, <-] (generator) ++(0, -0.5) -- ++(0, -0.5) -- ++(-0.5,0) node[left]{$V_{\mathrm{ref}_{\mathrm{1}}}, V_{\mathrm{ref}_{\mathrm{2}}}, V_{\mathrm{ref}_{\mathrm{3}}}, \dots, V_{\mathrm{ref}_{\mathrm{n}}}$};
		
	\end{tikzpicture}
}






	\caption{Reactive power controller disturbance assessment methodology (shown for voltage droop control)}
	\label{fig:smib-vref-change-diagram}
\end{figure}

In wide area studies, only reference step tests are applied, since actual voltage disturbances are best assessed through the application of real faults, machine trips and \ac{TOV} shunt application, all of which is assessed under s5.2.5.5.

