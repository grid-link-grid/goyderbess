\ac{PLR} tests assess the ability of the generator to ride through the angle change associated with a loss of a significant percentage of the load they are supplying into. While more relevant for traditional synchronous machines supplying the majority of the load to large load centres or industrial customers, an analogous \ac{PLR} test can also be performed in SMIB.

To perform this test, a load equivalent to 30\% of the maximum capacity of the generating system is attached at the connection point and the appropriate $V_{\mathrm{grid}_{\mathrm{initial}}}$ is  identified to achieve $V_{\mathrm{POC}_{\mathrm{initial}}}$ given the required initial $P_{\mathrm{POC}}$, $Q_{\mathrm{POC}}$, SCR and X/R conditions, noting that 30\% of the power flow is no longer being transferred to the slack bus. 

The test is then performed by initialising the system with the load in service, then switching it out of service at the intended disturbance time for the remainder of the simulation, as shown in Figure~\ref{fig:smib-plr-diagram}.


\begin{figure}[h]
	\centering
	\newcommand{\bushere}[3]{% length, text above, text below}
% Optional arguments do nto work in paths
%
% starting point; draw an edge and then two nodes
% save the position
coordinate(tmp)
% go up and do an edge down
++(0,#1) node[anchor=base, font=\footnotesize]{#2} edge[ultra thick] ++(0, {-2*#1})
% edges do not move the current point, go down to position the node
++(0,{-2*#1}) node[below]{#3}
% go back to where we started
(tmp)
}

\ctikzset{sources/fill=gray!20, resistors/fill=gray!20}
\resizebox{0.65\linewidth}{!}{ % Set width to \linewidth
\begin{tikzpicture}[semithick]% default line width
	% Buses and branches
	\draw (0,0)
	node[left, font=\footnotesize]{Generator} ++(1.5,0) \bushere{1.5}{Connection Point}{} coordinate(poc);
	\draw (poc)
	-- ++ (1,0) to[generic, l={$Z_{\text{grid}}$}, resistors/width=2] ++ (4,0)
	-- ++ (1,0)
	\bushere{1.5}{Infinite Bus}{} coordinate(infinite bus);
	% One load (start from the coord load, go up)
	\ctikzset{bipoles/border margin=0.5}% See manual section 3.1.2
	\draw (infinite bus) -- ++(1,0) node[vsourcesinshape, rotate=90]{} ++(0.5,0) node[right]{$V_{\text{grid}}$};
	\draw (poc) -- ++(-1,0) node[vsourcesinshape, rotate=90]{};
	% Capacitor
	\draw[red](poc) ++(0, -0.7) -- ++(0.7,0) -- ++(0,-0.5) to[R] ++(0,-1) node[ground]{};
	
\end{tikzpicture}
}





	\caption{PLR test application methodology}
	\label{fig:smib-plr-diagram}
\end{figure}