 \textbf{Voltage ‘droop’} = 4\% on 112.575 MVAr base

\begin{enumerate}
	\item The integrated resource system has plant capabilities and control systems sufficient to ensure that:
	\begin{enumerate}
		\item power system oscillations, for the frequencies of oscillation of the production unit against any other production unit or system, are adequately damped;
		\item operation of the integrated resource system does not degrade the damping of any critical mode of oscillation of the power system; and
		\item operation of the integrated resource system does not cause instability (including hunting of tap-changing transformer control systems) that would adversely impact other Registered Participants.
	\end{enumerate}
	\item The control systems used with this integrated resource system have:
	\begin{enumerate}
		\item for the purposes of disturbance monitoring and testing, permanently installed and operational, monitoring and recording facilities for key variables including each input and output; and
		\item facilities for testing the control system sufficient to establish its dynamic operational characteristics.
	\end{enumerate}
	\item The integrated resource system has facilities with a control system to regulate voltage, reactive power and power factor, with the ability to operate in any control mode and to switch between control modes, as shown in the plant voltage control strategy document
	\item The integrated resource system has a voltage control system that:
	\begin{enumerate}
		\item regulates voltage at the Connection Point to within 0.5\% of the setpoin;
		\item regulates voltage in a manner that helps to support network voltages during faults and does not prevent the NSP from achieving the requirements under clause S5.1a.3 and S5.1a.4 of the NER;
		\item allows the voltage setpoint to be continuously controllable in the range of at least 95\% to 105\% of the target voltage at [the Connection Point (as recorded in the connection agreement), without reliance on a tap-changing transformer and subject to the reactive power capability referred to in the performance standard under clause S5.2.5.1;
		\item has limiting devices to ensure that a voltage disturbance does not cause the production unit to trip at the limits of its operating capability.  The limiting devices:
			\begin{enumerate}
				\item do not detract from the performance of any power system stabiliser or power oscillation damping capability;  and
				\item are co-ordinated with all protection systems.
			\end{enumerate}
		
	\end{enumerate}
	\item The integrated resource system has a voltage control system that: 
	\begin{enumerate}
		\item with the integrated resource system connected to the power system, has settling times for active power, reactive power and voltage due to a step change of voltage setpoint or voltage at the connection point, of less than:
		\begin{enumerate}
			\item 5.0 s for a 5\% voltage disturbance with the integrated resource system connected to the power system, from an operating point where the voltage disturbance would not cause any limiting device to operate; and
			\item 7.5 s for a 5\% voltage disturbance with the integrated resource system connected to the power system, when operating into any limiting device from an operating point where a voltage disturbance of 2.5\% would just cause the limiting device to operate;
		\end{enumerate}
		\item for a 5\% step change in the voltage setpoint, has reactive power rise time, of less than 2 s;
		\item has power oscillation damping capability with sufficient flexibility to enable damping performance to be maximised with characteristics as described in paragraph (7).
	\end{enumerate}
	\item A reactive power or power factor control system provided under paragraph (3) will: 
	\begin{enumerate}
		\item regulate reactive power or power factor at the Connection Point, to within: 
		\begin{enumerate}
			\item for a integrated resource system operating in reactive power mode, 2\% of the generating system’s rating (expressed in MVAr);   
			\item for a integrated resource system operating in power factor mode, a power factor equivalent to 2\% of the integrated resource system's rating (expressed in MVAr); 
		\end{enumerate}
		\item allow the reactive power or power factor setpoint to be continuously controllable across the reactive power capability range established under the performance standard under clause S5.2.5.1; and 
		\item with the integrated resource system connected to the power system, and for a step change in setpoint of at least 50\% of the reactive power capability agreed with AEMO and the NSP under clause S5.2.5.1 of the NER, or a 5\% voltage disturbance at the location agreed under subparagraph (i):
		\begin{enumerate}
			\item have settling times for active power, reactive power and voltage of less than 5.0 s from an operating point where the voltage disturbance would not cause any limiting device to operate; and 
			\item have settling times for active power, reactive power and voltage of less than 7.5 s when operating into any limiting device from an operating point where a voltage disturbance of 2.5\% would just cause the limiting device to operate. 
		\end{enumerate}
	\end{enumerate}
\end{enumerate}
